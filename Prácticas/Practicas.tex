\documentclass[12pt,spanish]{article}
\usepackage[spanish]{babel}
\usepackage{graphicx}
\usepackage{color}
\usepackage{xcolor}
\usepackage{colortbl}
\usepackage{amsthm,thmtools}
\usepackage{dirtytalk}
\usepackage{multirow}
\usepackage{amsmath}
\usepackage{subcaption}
\usepackage{adjustbox}
\usepackage{amsmath}
\usepackage{centernot}
\usepackage{mathtools}
\usepackage{multirow}
\usepackage[hidelinks]{hyperref}
\usepackage{caption}
\usepackage{eurosym} % para el euro
\usepackage{amsthm}
\usepackage{multicol}
\usepackage{float}
\usepackage{amsfonts}
\usepackage{titling}
\usepackage{soul}
\usepackage{listings}
\usepackage{array}
\usepackage{tikz}
\usetikzlibrary{shapes.geometric, arrows, chains, calc,positioning,fit,decorations.pathreplacing}
\usepackage[framemethod=tikz]{mdframed}

\graphicspath{ {../img/}}
\selectlanguage{spanish}
\usepackage[utf8]{inputenc}
\usepackage{graphicx}
\usepackage[a4paper,left=3cm,right=2cm,top=2.5cm,bottom=2.5cm]{geometry}

\newenvironment{solution}{
	\par
	\textbf{Solución}
	\par
	\begin{center}
}
{
	\end{center}
}

\lstset{
  breaklines=true,
  postbreak=\mbox{\textcolor{red}{$\hookrightarrow$}\space},
}


\title{Servidores Web de Altas Prestaciones}
\setlength{\droptitle}{10em}
\author{Carlos Sánchez Páez}

\makeindex
\begin{document}
\definecolor{light-gray}{gray}{0.95}
\lstset{columns=fullflexible,basicstyle=\ttfamily}
\surroundwithmdframed[
  hidealllines=true,
  backgroundcolor=light-gray,
  innerleftmargin=0pt,
  innertopmargin=0pt,
  innerbottommargin=0pt]{lstlisting}


\begin{titlepage}

 \newlength{\centeroffset}
 \setlength{\centeroffset}{-0.5\oddsidemargin}
 \addtolength{\centeroffset}{0.5\evensidemargin}
 \thispagestyle{empty}

 \noindent\hspace*{\centeroffset}
 \begin{minipage}{\textwidth}

  \centering
  \includegraphics[width=0.9\textwidth]{logo_ugr.jpg}\\[1.4cm]

  \textsc{ \Large Servidores Web de Altas Prestaciones\\[0.2cm]}
  \textsc{GRADO EN INGENIERÍA INFORMÁTICA}\\[1cm]

  {\Huge\bfseries Prácticas resueltas \\}
 \end{minipage}

 \vspace{1.5cm}
 \noindent\hspace*{\centeroffset}
 \begin{minipage}{\textwidth}
  \centering

  \textbf{Autor}\\ {Carlos Sánchez Páez}\\[2.5ex]
  \includegraphics[width=0.4\textwidth]{etsiit_logo.png}\\[0.1cm]
  \vspace{1.5cm}
  \includegraphics[width=0.15\textwidth]{atc.jpg}\\[0.1cm]
  \vspace{1cm}
  \textsc{Escuela Técnica Superior de Ingenierías Informática y de Telecomunicación}\\
  \vspace{1cm}
  \textsc{Curso 2019-2020}
 \end{minipage}
\end{titlepage}
\thispagestyle{empty}
\newpage
\tableofcontents{}
\newpage

\section{Práctica 1}

En esta práctica configuraremos dos máquinas virtuales (\textit{M1} y \textit{M2}). También crearemos una interfaz \emph{host-only} para que las máquinas puedan conectarse entre ellas. \\

En esta guía configuraremos la interfaz sólo anfitrión manualmente en una máquina. En la otra dejaremos que el asistente de instalación lo haga por nosotros.

\begin{enumerate}
	\item Comenzaremos creando las máquinas virtuales desde VirtualBox. Las proveeremos de al menos 512MB de RAM y 10GB de disco duro.
	\item Descargamos la imagen ISO de Ubuntu Server 18.04 y la montamos en la unidad de disco de la primera máquina.
	\item Arrancamos la máquina e iniciamos el asistente de instalación. Establecemos nuestro nombre de usuario de GitHub como \emph{username} y \emph{m1} como nombre del servidor. La clave será \emph{Swap1234}
	\item Cuando termine la instalación apagamos la máquina.
	\item En VirtualBox abrimos el administrador de redes sólo antitrión (dentro del menú Archivo).
	\item Creamos un adaptador y activamos DHCP para que asigne una IP a nuestra máquina.
	\begin{figure}[H]
		\centering
		\includegraphics[scale=0.65]{/p1/host_only.png}
	\end{figure}
	\item Vamos a los ajustes de red de la máquina y ''conectamos'' el adaptador que acabamos de crear.
	\begin{figure}[H]
		\centering
		\includegraphics[scale=0.65]{/p1/host_only_2.png}
	\end{figure}
	\item Arrancamos la máquina. Ejecutamos el siguiente comando para ver las interfaces conectadas:
	\begin{lstlisting}
		m1> sudo ifconfig -a
	\end{lstlisting}
	\begin{figure}[H]
		\centering
		\includegraphics[scale=0.65]{/p1/ifconfig_noip.png}
	\end{figure}
	\item Vemos que tenemos una nueva interfaz (\emph{enp0s8}) pero no tiene IP asignada. Para arreglar esto usaremos \emph{netplan}.
	\item Creamos un archivo de configuración para ella:
	\begin{lstlisting}
		m1> sudo nano /etc/netplan/host-only.yaml
	\end{lstlisting}
	\item Introducimos este contenido en el archivo. Debemos usar espacios en vez de tabuladores.
	\begin{lstlisting}
		network:
		  version: 2
			renderer: networkd
			ethernets:
			  enp0s8:
				  dhcp4: true
	\end{lstlisting}
	\item Guardamos los cambios y los aplicamos con el comando
	\begin{lstlisting}
		m1> sudo netplan apply
	\end{lstlisting}
	\item Ejecutamos \emph{sudo ifconfig -a} y comprobamos que ya tenemos IP asignada:
	\begin{figure}[H]
		\centering
		\includegraphics[scale=0.65]{/p1/ifconfig.png}
	\end{figure}
	\item Instalamos la pila LAMP:
	\begin{lstlisting}
	m1> sudo apt install -y openssh-server openssh-client apache2 mysql-server mysql-client
	\end{lstlisting}
	\item Pasamos ahora a la configuración de \emph{m2}. Para ello seguimos los pasos anteriores. Como el adaptador sólo anfitrión ya está creado, el asistente de instalación lo configurará automáticamente. Lo único que tenemos que hacer es ''conectarlo'' a la máquina como hicimos antes.
	\item Cuando termine la instalación, ejecutamos el comando anterior para instalar la pila LAMP.
	\item Probamos la conexión SSH conectando las máquinas entre sí. Para facilitar las conexiones podemos guardar las IPs en variables:
	\begin{figure}[H]
		\centering
		\includegraphics[scale=0.5]{/p1/ip_variable.png}
	\end{figure}
	\begin{figure}[H]
		\centering
		\includegraphics[scale=0.5]{/p1/ssh_ok.png}
	\end{figure}
	\item Creamos en \emph{M2} un documento HTML (\emph{/var/www/html/ejemplo.html}) con el siguiente contenido:
	\begin{lstlisting}
		<HTML>
		<BODY>
		Web de ejemplo de <nombre usuario> para SWAP
		</BODY>
		</HTML>
	\end{lstlisting}
	\item Hacemos \emph{curl} desde la otra máquina y comprobamos que recibimos la respuesta esperada:
	\begin{figure}[H]
		\centering
		\includegraphics[scale=0.5]{/p1/curl_ok.png}
	\end{figure}
\end{enumerate}

\section{Práctica 2}

En esta práctica clonaremos información entre las máquinas.

\begin{enumerate}
	\item Comenzamos copiando un directorio de m1 a la carpeta de usuario de \emph{m2}:
	\begin{lstlisting}
	m1 > mkdir test ; touch test/file1.txt test/file2.txt
	m1 > scp -r test $M2_IP:~
	\end{lstlisting}
	Ejecutamos \emph{ls -lR} en \emph{M2} para ver el resultado:
	\begin{figure}[H]
		\centering
		\includegraphics[scale=0.5]{/p2/scp_ok.png}
	\end{figure}
	\item Para no tener que introducir la contraseña en cada conexión SSH configuraremos la autenticación mediante clave público-privada RSA:
	\begin{lstlisting}<HTML>
	m2 > ssh-keygen -b 4096 -t rsa
	\end{lstlisting}
	Pulsamos Enter tres veces.
	\item Copiamos la clave a M1:
	\begin{lstlisting}
	m2 > ssh-copy-id $M1_IP
	\end{lstlisting}
	Introducimos la clave \emph{Swap1234}
	\item Probamos a realizar la conexión. Veremos que no nos pide la clave.
	\begin{figure}[H]
		\centering
		\includegraphics[scale=0.65]{/p2/ssh_ok.png}
	\end{figure}
	\item Por último programamos una tarea de clonación que se ejecutará cada dos horas. De esta forma el contenido de M1 se replicará en M2 cada dos horas. Usaremos \emph{crontab} para ello:
	\item Lo primero que debemos hacer es configurar la clave público privada por ssh en M1 para que no se pida en cada copia (igual que en el paso anterior).
	\item Después damos permisos a nuestro usuario para escribir en el directorio:
	\begin{lstlisting}
	m2 > sudo chown $USER:$USER -R /var/www
	\end{lstlisting}
	\item Configuramos el clonado:
	\begin{lstlisting}
	m1 > sudo nano /etc/crontab
	\end{lstlisting}
	Añadimos la siguiente línea:
	\begin{lstlisting}
	00 */12 * * * csp98 scp -r /var/www/ 192.168.56.103:/var/
	\end{lstlisting}
	\item Si queremos probar el funcionamiento del comando podemos establecer una periodicidad menor, crear un archivo en M1 y ver que se replica en M2.
\end{enumerate}

\section{Práctica 3}
En esta práctica configuraremos un balanceador de carga (\emph{nginx}).
\begin{enumerate}
	\item Comenzamos creando una nueva máquina (M3) e instalando Ubuntu Server. Le añadiremos también el adaptador \emph{host-only}.
	\item Almacenamos el alias de ambas IPs para ahorrar tiempo en futuros comandos:
		\begin{lstlisting}
		m3 > echo "M1_IP=192.168.56.100" >> .bashrc ; source .bashrc
		m3 > echo "M2_IP=192.168.56.103" >> .bashrc ; source .bashrc
		\end{lstlisting}
	\item Instalamos nginx y lo lanzamos:
		\begin{lstlisting}
		m3 > sudo apt update ; sudo apt install nginx
		m3 > sudo systemctl start nginx
		\end{lstlisting}
	\item Como solo queremos que funcione como balanceador de carga, desactivamos la funcionalidad de servidor web. Para ello editamos el archivo de configuración y comentamos la siguiente línea:
		\begin{lstlisting}
		m3 > sudo nano /etc/nginx/nginx.conf
		\end{lstlisting}
			\begin{lstlisting}
			# include /etc/nginx/sites-enabled/*
			\end{lstlisting}
		\item Configuramos el \emph{upstream}, máquinas entre las que se dividirá el tráfico:
		\begin{lstlisting}
		m3 > sudo nano /etc/nginx/conf.d/default.conf
		\end{lstlisting}
		Insertamos el siguiente contenido:
		\begin{lstlisting}
		upstream servidoresSWAP{
			server 192.168.56.100;
			server 192.168.56.103;
		}

		server{
			listen 80;
			server_name balanceador;
			access_log /var/log/nginx/balanceador.access.log;
			error_log /var/log/nginx/balanceador.error.log;
			root /var/www/;
			location /
			{
				proxy_pass http://servidoresSWAP;
				proxy_set_header Host $host;
				proxy_set_header X-Real-IP $remote_addr;
				proxy_set_header X-Forwarded-For $proxy_add_x_forwarded_for;
				proxy_http_version 1.1;
				proxy_set_header Connection "";
			}
		}
		\end{lstlisting}
		\item Reiniciamos el servicio para que los cambios surtan efecto:
		\begin{lstlisting}
		m3 > sudo service nginx restart
		\end{lstlisting}
		\item Arrancamos M1 y M2. Para diferenciar a cual de ellas se envía la petición a través del balanceador de carga, modificaremos el archivo \emph{/var/www/html/ejemplo.html} de cada una:
		\textcolor{red}{IMÁGENES DE AMBOS HTML}
		\item Consultamos la IP de M3 y entramos a \emph{192.168.56.104/ejemplo.html} desde el navegador de nuestro host:
		\begin{lstlisting}
		m3 > ip addr | grep 192
		\end{lstlisting}
		\item Recargamos varias veces la página y podremos ver como la petición la atiende un servidor u otro.
		\textcolor{red}{IMÁGENES DE AMBAS WEBS}
		\item Probaremos ahora a repartir la carga de forma desigual. M1 atenderá el doble de peticiones que M2. Para ello editamos el siguiente archivo de configuración y añadimos la directiva \emph{weight}:
		\item Configuramos el \emph{upstream}, máquinas entre las que se dividirá el tráfico:
		\begin{lstlisting}
		m3 > sudo nano /etc/nginx/conf.d/default.conf
		\end{lstlisting}
		\begin{lstlisting}
		upstream servidoresSWAP{
			server 192.168.56.100 weight=2;
			server 192.168.56.103 weight=1;
		}
		\end{lstlisting}
		\item Reiniciamos el servicio y probamos a acceder con el navegador a la IP anterior y refrescar la página.
		\item Tambien podemos especificar políticas para mantener la sesión:
		\begin{itemize}
			\item \textbf{Keep-alive}: las peticiones se enviarán al mismo servidor durante $n$ segundos.
			\begin{lstlisting}
			upstream servidoresSWAP{
				server 192.168.56.100;
				server 192.168.56.103;
				keepalive <n>;
			}
			\end{lstlisting}
			\item \textbf{IP-hash}. Una vez que un servidor atiende una IP, la atenderá siempre. No se recomienda su uso (no se balancea la carga de forma equitativa).
			\begin{lstlisting}
			upstream servidoresSWAP{
				ip_hash;
				server 192.168.56.100;
				server 192.168.56.103;
			}
			\end{lstlisting}
		\end{itemize}


	\end{enumerate}






\end{document}
