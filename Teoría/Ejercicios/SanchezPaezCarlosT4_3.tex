\documentclass[12pt,spanish]{article}
\usepackage[spanish]{babel}
\usepackage{graphicx}
\usepackage{color}
\usepackage{xcolor}
\usepackage{colortbl}
\usepackage{amsthm,thmtools}
\usepackage{dirtytalk}
\usepackage{multirow}
\usepackage{amsmath}
\usepackage{subcaption}
\usepackage{adjustbox}
\usepackage{amsmath}
\usepackage{centernot}
\usepackage{mathtools}
\usepackage{multirow}
\usepackage[hidelinks]{hyperref}
\usepackage{caption}
\usepackage{eurosym} % para el euro
\usepackage{amsthm}
\usepackage{multicol}
\usepackage{float}
\usepackage{amsfonts}
\usepackage{titling}
\usepackage{soul}
\usepackage{listings}
\usepackage{array}
\usepackage{tabulary}
\usepackage{tikz}
\usetikzlibrary{shapes.geometric, arrows, chains, calc,positioning,fit,decorations.pathreplacing}
\usepackage[framemethod=tikz]{mdframed}

\graphicspath{ {../../img/}}
\selectlanguage{spanish}
\usepackage[utf8]{inputenc}
\usepackage{graphicx}
\usepackage[a4paper,left=3cm,right=2cm,top=2.5cm,bottom=2.5cm]{geometry}

\newenvironment{solution}{
	\par
	\textbf{Solución}
	\par
	\begin{center}
}
{
	\end{center}
}

\lstset{
  breaklines=true,
  postbreak=\mbox{\textcolor{red}{$\hookrightarrow$}\space},
}


\title{Servidores Web de Altas Prestaciones}
\setlength{\droptitle}{10em}
\author{Carlos Sánchez Páez}

\makeindex
\begin{document}
\definecolor{light-gray}{gray}{0.95}
\lstset{columns=fullflexible,basicstyle=\ttfamily}
\surroundwithmdframed[
  hidealllines=true,
  backgroundcolor=light-gray,
  innerleftmargin=0pt,
  innertopmargin=0pt,
  innerbottommargin=0pt]{lstlisting}


\begin{titlepage}

 \newlength{\centeroffset}
 \setlength{\centeroffset}{-0.5\oddsidemargin}
 \addtolength{\centeroffset}{0.5\evensidemargin}
 \thispagestyle{empty}

 \noindent\hspace*{\centeroffset}
 \begin{minipage}{\textwidth}

  \centering
  \includegraphics[width=0.9\textwidth]{logo_ugr.jpg}\\[1.4cm]

  \textsc{ \Large Servidores Web de Altas Prestaciones\\[0.2cm]}
  \textsc{GRADO EN INGENIERÍA INFORMÁTICA}\\[1cm]

  {\Huge\bfseries Balanceadores hardware \\}
 \end{minipage}

 \vspace{1.5cm}
 \noindent\hspace*{\centeroffset}
 \begin{minipage}{\textwidth}
  \centering

  \textbf{Autor}\\ {Carlos Sánchez Páez}\\[2.5ex]
  \includegraphics[width=0.4\textwidth]{etsiit_logo.png}\\[0.1cm]
  \vspace{1.5cm}
  \includegraphics[width=0.15\textwidth]{atc.jpg}\\[0.1cm]
  \vspace{1cm}
  \textsc{Escuela Técnica Superior de Ingenierías Informática y de Telecomunicación}\\
  \vspace{1cm}
  \textsc{Curso 2019-2020}
 \end{minipage}
\end{titlepage}
\thispagestyle{empty}
\newpage


En esta tarea compararemos dos balanceadores de carga hardware, en concreto el TP-LINK TK-R470T+ (\href{https://www.tp-link.com/es/business-networking/load-balance-router/tl-r470t+/}{https://www.tp-link.com/es/business-networking/load-balance-router/tl-r470t+/}) y el TL-ER5120 (\href{https://www.tp-link.com/es/business-networking/load-balance-router/tl-er5120/}{https://www.tp-link.com/es/business-networking/load-balance-router/tl-er5120/}).\\

\begin{figure}[H]
	\footnotesize
	\begin{tabular}{|p{1.2cm}|p{1.1cm}|p{1.1cm}|p{1.1cm}|p{1.3cm}|p{1.3cm}|p{1.2cm}|p{1.7cm}|p{1.1cm}|p{1.2cm}|}
		\hline
		Modelo & RAM & Flash & Puertos fijos WAN & Puertos fijos LAN & Puertos intercambiables WAN/LAN &  Tipo de puerto & Conexiones concurrentes & Soporte IPv6 &  Precio \\
		\hline
		TL-R470T+ & 128MB & 16MB & 1 & 1 & 3 & Fast Ethernet & 10.000 & No & 37 euros \\
		\hline
		TL-ER5120 & 256MB & 32MB & 1 & 1 & 3 & Gigabit & 150.000  & Sí & 140 euros\\
		\hline
	\end{tabular}
\end{figure}


\end{document}
